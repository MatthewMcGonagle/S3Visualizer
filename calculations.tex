\documentclass{article}

\usepackage{amsmath, amssymb, mathtools}

\title{Calculation for Finding Intersection of Geodesic Ray in \(S^3\) with Euclidean Ball}
\author{Matthew McGonagle}

\begin{document}
\maketitle

    We are given \(p\in S^3\) with \(v \in T_p S^3\) a unit vector. Furthermore, we are given a ball \(B_r(q)\subset S^3\) such that \(B_r(q) = \{\|x - q\|<r\}\) where \(\|x\|\) is the ordinary Euclidean vector norm. We wish to find if the forward geodesic ray in \(S^3\), starting at \(p\) and in the direction of \(v\), will intersect \(B_r(q)\). 

    Note that the geodesic ray is parameterized by \(x(t) = \cos(t) p + \sin(t) v\) where \(t\geq 0\). So, finding the intersection points of \(x(t)\) with \(B_r(q)\) is equivalent to finding the non-negative solutions \(t\geq 0\) to 
\[\|\cos(t) p +\sin(t) v - q\|^2 = r^2.\]
Using that \(\{p,v\}\) is an orthonormal set of vectors in Euclidean space and that \(q\) is a unit vector, we obtain that
\[ 2 - 2\langle p,q \rangle \cos(t) - 2 \langle v, q\rangle \sin(t) = r^2.\]

So we have intersection points if and only if there are solutions to 
\[\langle p,q \rangle \cos(t) + \langle v, q\rangle \sin(t) = 1 - \frac{r^2}{2}. \]
This is more informatively expressed as a dot product of vectors in \(\mathbb R^2\):
\[\begin{pmatrix} \langle p,q\rangle \\ \langle v, q \rangle\end{pmatrix} \cdot \begin{pmatrix} \cos(t) \\ \sin(t) \end{pmatrix} = |1 - \frac{r^2}{2}|. \]
Since the minimum and maximum of the dot product on the left hand side is \(\pm \|(\langle p,q\rangle, \langle v, q\rangle)\|\), we see that there is are intersection points if and only if \(\sqrt{\langle p,q\rangle^2 + \langle v,q\rangle^2} \geq 1 - r^2/2.\)

Now let \((a,b)\) be the normalization of \((\langle p,q\rangle, \langle v, q\rangle)\), and let \(D = (1-r^2/2)(\langle p,q\rangle^2 + \langle v, q\rangle^2)^{-1/2}\). So our equation becomes
\[\begin{pmatrix} a \\ b \end{pmatrix} \cdot \begin{pmatrix} cos(t) \\ sin(t) \end{pmatrix} = D.\] 
Note that solutions exist if and only if \( |D|\leq 1\).


We now get two sets of equations. We must compare the solutions of each set to find the smallest solution \(t>0\). The first set of equations is
\[\begin{pmatrix} \cos(t) \\ \sin(t)\end{pmatrix} = D \begin{pmatrix} a \\ b \end{pmatrix} + \sqrt{1-D^2}\begin{pmatrix} -b \\ a \end{pmatrix}.\]
The second set of equations is
\[\begin{pmatrix} \cos(t) \\ \sin(t)\end{pmatrix} = D \begin{pmatrix} a \\ b \end{pmatrix} - \sqrt{1-D^2}\begin{pmatrix} -b \\ a \end{pmatrix}.\]

The first set of equations may be solved by finding the solutions \(t>0\) of 
\[\tan(t) = \frac{D b + \sqrt{1-D^2} a}{D a - \sqrt{1-D^2} b}, \]
that are in the correct quadrant. For this, one must check the signs of \( D b + \sqrt{1-D^2} a\) and \(D a - \sqrt{1-D^2}b\).

The second set of equations may be solved by finding the solutions \(t>0\) of
\[\tan(t) = \frac{D b - \sqrt{1 - D^2} a}{D a + \sqrt{1-D^2} b},\]
that are in the correct quadrant. Thus one must check the signs of \(D b - \sqrt{1 - D^2} a\) and \(D a + \sqrt{1-D^2} b\).

Finally, one must compare the solutions of both equations to find the smallest possible solutions \(t>0\). This gives the true intersection time \(t>0\) of the light ray.
\end{document}
